\subsection{\textbf{RQ2:} What unmet needs impact the difficulty of a conflict resolution?}\label{RQ2}

The perception of merge conflict difficulty and the actual hurdles of resolving them are distinct, and at times separated by a gap in knowledge or awareness that prevent practitioners from accurately gauging the difficulty of resolving merge conflicts.
We asked interview participants a series of open-ended questions regarding unmet needs, the impact of unmet needs on understandability, and their impact on resolving merge conflicts in real-world situations.

Using results from the interview to inform the construction of the survey, we asked survey participants to rate how much each of 10 unmet needs affected their ability to resolve merge conflicts.
We received 141 responses using a 5-point Likert scale indicating the degree of effect on resolution difficulty (1 being \textit{Not at all}, 3 being \textit{A moderate amount}, and 5 being \textit{A great deal}).

Results of the survey are presented in Table~\ref{survey_res_diffs}.
We present and discuss in detail the top 4 unmet needs based on mean score.
The top 4 unmet needs all relate to practitioners' understanding and their expertise in resolving conflicts. 

All of the unmet needs in Table~\ref{survey_res_diffs} can be considered important to practitioners when resolving merge conflicts, since each received a mean score of at least 3.03 on the 5-point Likert scale.
When analyzed across demographic categories, we find that practitioners uniformly agree with this assessment except for N10 when examined across open-source/closed-source practitioners.
We further discuss these demographic differences for N10 in Section~\ref{oss_vs_closed_tool_support}.

\subsubsection{\underline{Understanding Code Involved in a Merge Conflict}}
We found that practitioners consider the ability to understand the code involved in a merge conflict to be the unmet need with the highest impact on resolution (N1).
For instance, when working with two conflicting commits of different sizes, P1 said:
\begin{displayquote}
\textit{``You focus on understanding the small change, not the big one. It's easier to understand... get the small change to go with the flow of the bigger change.''}
\end{displayquote}
Prior work on workspace awareness and codebase visualizations~\cite{palantir}\cite{lanza_manhattan} have similarly found that knowledge of code is an important factor in preventing the occurrence of merge conflicts, or reducing the difficulty in resolving them.

\begin{table*}[!htbp]
\renewcommand{\arraystretch}{1.3}
\caption{Improvements for Practitioner Toolsets}
\label{survey_tool_needs}
\centering
%\begin{tabularx}{0.9\textwidth}{@{}r|*{6}{C}c@{}}
\begin{tabularx}{0.852\textwidth}{>{\rowmac}c | >{\rowmac}l | *5{>{\rowmac}c} | *2{>{\rowmac}c}<{\clearrow}}

\toprule
	Improvement & Description & 1 & 2 & 3 & 4 & 5 & Mean & Median\\
\midrule
	\setrow{\bfseries}I1 & Better usability & 6 & 17 & 32 & 48 & 16 & 3.43 & 4\\
	\setrow{\bfseries}I2 & Better ways of filtering out less relevant information & 8 & 15 & 32 & 48 & 16 & 3.41 & 4\\
	\setrow{\bfseries}I3 & Better ways of exploring project history & 7 & 21 & 36 & 39 & 16 & 3.30 & 3\\
	\setrow{\bfseries}I4 & Better graphical presentation of information & 13 & 26 & 26 & 37 & 16 & 3.14 & 3\\
	I5 & Better transparency in tool functionality/operations & 16 & 36 & 24 & 40 & 3 & 2.82 & 3\\
	I6 & Better terminology that is more consistent with my other tools & 23 & 41 & 32 & 15 & 8 & 2.53 & 2\\
	\bottomrule
\end{tabularx}
\end{table*}

\subsubsection{\underline{Expertise in Area of Merge Conflict}}
Knowledge is a key component of practitioner's needs when resolving merge conflicts, but along with that general knowledge is a need for expertise in the specific areas of code involved in a conflict.
Practitioners recognize this need as having a sizable effect on their ability to resolve a merge conflict (N2).

This result aligns with the goals of the TIPMerge tool~\cite{CostaSarma}, which seeks to locate experts that are best suited to resolve conflicts in a particular area of code.
Since practitioners indicate that this is the second most impactful unmet need, more focus is needed on developing team processes, policies, or tools to leverage the advantages of expertise.

\subsubsection{\underline{Amount of Information About Conflicting Code}}
Understanding the code is important to resolving a merge conflict, but knowledge of factors such as change history, surrounding code, and design decisions allow greater awareness of the causes that lead to a merge conflict.
Our results show that participants selected \textit{the amount of information you have about the conflicting code} as the third most impactful unmet need (N3).

During our interview, P7 was asked about information needs during merge conflicts and said:
\begin{displayquote}
\textit{``I can imagine that it's harder to merge code when you're merging in some legacy code which was written by some [former contributors] thirty years ago. But if you're a young team, and everybody who wrote the code is still a part of the team, it's easier.''}
\end{displayquote}

This unmet need, along N1 and N2, illustrate a theme of both knowledge and awareness of information that practitioners see as the biggest unmet needs when working with merge conflicts.
This theme leads us to conclude that further support through processes and tools are needed, and that research to lower this burden is worthwhile.

\subsubsection{\underline{Research Bias}}
Based on prior work stating the value of commit messages~\cite{yamauchi2014clustering}\cite{hindle2009automatic}\cite{cortes2014automatically}\cite{hattori2008nature}, we expected commit messages to be a helpful tool in merge conflict resolution.
However, we found that \textit{informativeness of commit messages} (N8) was toward the bottom of the rankings compared to other unmet needs.

With a mean of 3.07 on a 5-point Likert scale, N8 has \textit{a moderate amount} of effect on the difficulty of resolving a merge conflict according to survey participants.
But considering the position of 8th in a list of 10 unmet needs, commit message quality appears to receive a disproportionate amount of focus by the research community.



\subsubsection{\underline{Open-Source vs. Closed-Source Tool Support}}
\label{oss_vs_closed_tool_support} 

We found that practitioners who focus on open-source software development consider \textit{tool support for examining development history} (N10) to be the 3rd highest unmet need (mean: 3.60).
Whereas, practitioners who focus on closed-source software development consider it to be the 10th, and least, impactful unmet need (mean: 2.86).
This trend was also evident in our interviews, with P8 stating:

\begin{displayquote}
\textit{``I'm often dealing with code other people wrote... Nobody can review every pull request that goes in. So now I have to go back and do some archaeology to find out what's going on. Code is much easier to write than read.''}
\end{displayquote}

This result suggests that either history exploration in open-source projects is a more difficult task, or that tools are better at supporting history exploration use cases in closed-source development environments.

