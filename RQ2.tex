\subsection{RQ2: What factors impact the difficulty of a conflict resolution?}\label{RQ2}

\todo{talk about resolution difficulties in interviews and how survey answer options were formed}
\todo{get std deviations}

Our survey results show that three of the factors we investigated have an effect on the difficulty of the process of resolving merge conflicts: \textit{The amount of information you have about the conflicting code}(mean: 3.62) \textit{Your expertise in the area of code with the merge conflict} (mean: 3.72), and \textit{How easy it is to understand the code involved in the merge conflict} (mean: 3.89). 

Participants also rated the following factors, but their mean response values place them in the neutral category:
\begin{enumerate}
	\item \textit{How well tools present information in an understandable way} (mean: 3.48)
	\item \textit{Changing assumptions within the code} (3.30)
	\item \textit{Complexity of the project structure} (3.18)
	\item \textit{Trustworthiness of tools} (3.12)
	\item \textit{Informativeness of commit messages} (3.07)
	\item \textit{Project culture} (3.04)
	\item \textit{Tool support for examining development history} (3.03)
\end{enumerate}

No factor in this category had an average below 3.03, so there were no factors tested that have little to no effect on the difficulty of a merge conflict resolution.

Based on prior work stating the value of commit messages \cite{yamauchi2014clustering} \cite{hindle2009automatic} \cite{cortes2014automatically} \cite{hattori2008nature}, we expected \textit{Informativeness of commit messages} to have a larger impact on the difficulty of a merge conflict resolution. One possible explanation for this discrepancy between the literature and perceptions is that commit messages provide a good source of metadata for researchers mining software repositories, but they are often too short to provide much insight to a software practitioner in need of information.