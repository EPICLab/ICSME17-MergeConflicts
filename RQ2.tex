\subsection{\textbf{RQ2:} What unmet needs impact the difficulty of a conflict resolution?}\label{RQ2}

The perception of merge conflict difficulty and the actual hurdles of resolving them are distinct, and at times separated by a gap in knowledge or awareness.
These gaps prevent practitioners from accurately gauging the difficulty before beginning to resolve a merge conflict.
We asked interview participants a series of open-ended questions regarding unmet needs, the impact of unmet needs on understandability, and their impact on resolving merge conflicts in real-world situations.

Using results from the interview to inform the construction of the survey, we asked survey participants to rate how much each of 10 needs affected their ability to resolve merge conflicts.
We received 141 responses using a 5-point Likert scale indicating the degree of effect on resolution difficulty (1 being \textit{Not at all}, 3 being \textit{A moderate amount}, and 5 being \textit{A great deal}).

Results of the survey are presented in Table~\ref{survey_res_diffs}.
We present and discuss in detail the top 4 unmet needs based on mean score.
The top 4 needs all relate to practitioners' understanding and their expertise in resolving conflicts. 

All of the needs in Table~\ref{survey_res_diffs} can be considered important to practitioners when resolving merge conflicts, since each received a mean score of at least 3.03 on the 5-point Likert scale.
However, practitioners primarily working on closed-source development differed in perceptions of \textit{tool support for examining development history} (N10).
We further discuss this demographic difference in Section~\ref{oss_vs_closed_tool_support}.

\begin{table*}[!htbp]
\renewcommand{\arraystretch}{1.3}
\caption{Improvements for Practitioner Toolsets}
\label{survey_tool_needs}
\centering
%\begin{tabularx}{0.9\textwidth}{@{}r|*{6}{C}c@{}}
\begin{tabularx}{0.852\textwidth}{>{\rowmac}c | >{\rowmac}l | *5{>{\rowmac}c} | *2{>{\rowmac}c}<{\clearrow}}

\toprule
	Improvement & Description & 1 & 2 & 3 & 4 & 5 & Mean & Median\\
\midrule
	\setrow{\bfseries}I1 & Better usability & 6 & 17 & 32 & 48 & 16 & 3.43 & 4\\
	\setrow{\bfseries}I2 & Better ways of filtering out less relevant information & 8 & 15 & 32 & 48 & 16 & 3.41 & 4\\
	\setrow{\bfseries}I3 & Better ways of exploring project history & 7 & 21 & 36 & 39 & 16 & 3.30 & 3\\
	\setrow{\bfseries}I4 & Better graphical presentation of information & 13 & 26 & 26 & 37 & 16 & 3.14 & 3\\
	I5 & Better transparency in tool functionality/operations & 16 & 36 & 24 & 40 & 3 & 2.82 & 3\\
	I6 & Better terminology that is more consistent with my other tools & 23 & 41 & 32 & 15 & 8 & 2.53 & 2\\
	\bottomrule
\end{tabularx}
\end{table*}

\Subsubsection{Technical Aspects}
Among the top 4 needs, we find that three relate to technical needs that practitioners have when resolving merge conflicts.
The understandability of conflicting code is ranked as the most important need (N1), with both contextual information about the conflict (N3) and the way in which tools present relevant information (N4) also ranking toward the top of practitioners' concerns.

Version control systems allow for distributed development, but also create the situation in which merge conflicts can occur.
Practitioners often look at changes in isolation to deal with the technical complexities, but may overlook understanding the conflicting code in context of the larger codebase.
Decomposing the conflict into small changes makes it easier to understand, however, this can cause tunnel vision and lead to defects arising from the merge resolution.

Interview participants expressed similar concerns when discussing the difficulties of understanding large conflicts that span multiple files and complex blocks of code.
For instance, when working with two conflicting commits of different sizes, P1 said:
\begin{displayquote}
\textit{``You focus on understanding the small change, not the big one. It's easier to understand... get the small change to go with the flow of the bigger change.''}
\end{displayquote}

Additionally, even with knowledge and understanding about the particular code involved in a merge conflict, participants still require external knowledge about the conflicting code.
Understanding the rationale for specific design decisions, being aware of upcoming feature changes, and having access to change history can all reduce the potential for introducing bugs into the codebase while resolving a conflict.
During our interview, P7 was asked about information needs during merge conflicts and said:
\begin{displayquote}
\textit{``I can imagine that it's harder to merge code when you're merging in some legacy code which was written by some [former contributors] thirty years ago. But if you're a young team, and everybody who wrote the code is still a part of the team, it's easier.''}
\end{displayquote}

Prior work on workspace awareness and codebase visualizations~\cite{palantir}\cite{lanza_manhattan} have similarly found that knowledge of code and awareness of context are important factors in preventing the occurrence of merge conflicts, and reducing the difficulty in resolving them when they occur.
Both interview and survey participants indicated that these needs are important, and to some degree unmet, which suggests that further research is needed to improve both the presentation of information by tools, and lower the burden of comprehending any new information given by them.

\Subsubsection{Social Aspects}
Knowledge is a key component of practitioner's needs when resolving merge conflicts, but along with that general knowledge is a need for expertise in the specific areas of code involved in a conflict.
Practitioners recognize this need as having a sizable effect on their ability to resolve a merge conflict, and selected \textit{expertise in the area of conflicting code} (N2) as the second most important need.

Examining code artifacts, reviewing change history, and reading documentation help with understanding the code when they are present and well-maintained.
However, locating and maintaining these supporting documents is not always possible.
In fact, Forward et al.~\cite{forward2002documentation} conducted a survey of 48 software practitioners and found that 68\% either agreed or strongly agreed that documentation is always outdated.
When these gaps arise, practitioners compensate by consulting experts in the area of conflicting code instead.

This result aligns with the goals of the TIPMerge tool~\cite{CostaSarma}, which seeks to locate experts that are best suited to resolve conflicts in a particular area of code.
However, these tools have not made it into the hands of real-world practitioners, as evidenced by the lack of such tools in the list of top 10 merge tools (see Table~\ref{survey_toolset}).
Since practitioners indicate that this is the second most impactful need, more focus is needed on developing team processes, policies, or tools to leverage the expertise within teams.

\todo{Anita: Do we keep the sub-section title, Focus on Commit Messages?}
\Subsubsection{Focus on Commit Messages}
Based on prior work stating the value of commit messages~\cite{yamauchi2014clustering, hindle2009automatic, cortes2014automatically, hattori2008nature}, we expected commit messages to be a helpful tool in merge conflict resolution.
However, we found that \textit{informativeness of commit messages} (N8) was near the bottom of the rankings compared to other needs.
With a mean of 3.07 on a 5-point Likert scale, N8 has \textit{a moderate amount} of effect on the difficulty of resolving a merge conflict according to survey participants.
But considering the position of 8th in a list of 10 needs, commit message quality appears to receive a disproportionate amount of focus by the research community.

\Subsubsection{Open-Source vs. Closed-Source Needs}\label{oss_vs_closed_tool_support} 
It is interesting to note that for needs N1-N8 there was no statistical difference between practitioners focused on open-source and those focused on closed-source development when it comes to their conflict resolution needs.
We found that practitioners who focus on open-source software development consider \textit{tool support for examining development history} (N10) to be the 3rd highest unmet need (mean: 3.60).
Whereas, practitioners who focus on closed-source software development consider it to be the least impactful unmet need (mean: 2.86).
A Wilcoxon Signed-Ranks test indicated that open-source practitioners' ranking of N10 were statistically higher than closed-source practitioners' rankings of N10 (W = 1192.5, $p$-value = 0.02006).

This trend was also evident in our interviews, with P8 stating:

\begin{displayquote}
\textit{``I'm often dealing with code other people wrote... Nobody can review every pull request that goes in. So now I have to go back and do some archaeology to find out what's going on. Code is much easier to write than read.''}
\end{displayquote}

This result suggests that either history exploration in open-source projects is a more difficult task, or that tools are better at supporting history exploration in closed-source development environments.

