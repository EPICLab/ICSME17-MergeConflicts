\subsection{RQ3: Do developer tools meet developer needs for merge conflicts?}

\todo{talk about tool needs in interviews and how survey answer options were formed}

Our survey suggests that regardless of gender, developer role, experience level, project size, and source distribution model, software practitioners say that the following factors affect the difficulty of a merge conflict most:

\begin{itemize}
	\item \textit{Exploring project history}\\
	\item \textit{Filtering relevant info}\\
	\item \textit{Better usability}\\
\end{itemize}

People universally say that these have no effect:
\begin{itemize}
	\item \textit{Better terminology}\\
	\item \textit{Better transparency}\\
\end{itemize}

Female developers generally don’t see these as useful features. Five of six were rated as Not useful, while one was 50/50. This can be explained either by the small sample size (7 respondents), or by the fact that all tool features in the survey were derived from the interviews. Since all interview participants were male, these options may not connect with female software practitioners.

Tool transparency was not controversial by role or gender, but it was controversial by experience and project size.

Developers with 6-10 years of experience said terminology was useful. Nobody else thought it was useful.

Tool transparency for open vs closed source vs both. Both people might be annoyed about not being able to see some tool operations that are transparent in OSS tools. Groups on either side are "Domesticated"

Check tools people reported to see if closed source people use closed source tools.

Tool needs seem to be universal across roles. 


%\textbf{***Usability is important***}
%
%\textbf{***Most don't seem to care about terminology***}
%
%\textbf{***Better graphical presentation  of information seems to mattter more for less experienced devs***}
%
%\textbf{*** The most experienced groups want better wats to filter out less relevant information***}
%
%\textbf{***People care less about the transparency in operations***}