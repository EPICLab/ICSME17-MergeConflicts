\subsection{\textbf{RQ3:} Do developer tools meet developer needs for merge conflicts?}\label{RQ3}
Practitioners expect their merge tools to be easy to use, provide relevant information, and present that information in an understandable manner.
In order to understand which improvements practitioners value most, we asked interview participants open-ended questions about their merge conflict resolution process.

Our interviews informed the survey questions in which participants ranked 6 practitioner needs for tool improvements.
We received 119 responses using a 5-point Likert scale to indicate the usefulness each type of tool improvement (1 being \textit{Not Useful}, 3 being \textit{Moderately Useful}, and 5 being \textit{Essential}).

In addition, we also asked participants which tools they use during conflict resolution.
We received 105 different tools from 115 responses, though some were generic responses such as \textit{``my text editor''}.
We group these generic responses together where semantically similar meanings exist. 
Table~\ref{survey_toolset} lists the top 10 most common tools used by participants to resolve merge conflicts.

\begin{table}[!htbp]
\renewcommand{\arraystretch}{1.3}
\caption{Survey Participant Toolset (Top 10 tools)}
\label{survey_toolset}
\centering
\begin{tabularx}{0.45\textwidth}{@{}r|Cl@{}}
\toprule
Tool & \# Participants & Description\\
\midrule
Git	& 37 & Version Control System\\
Vim/vi & 17 & Text Editor\\
Text Editor (unspecified) & 14 & Text Editor\\
Git Diff & 11 & Diffing Tool\\
GitHub & 11 & Website\\
Eclipse & 10 & IDE\\
KDiff3 & 9 & Diff \& Merge\\
Meld & 8 & Diff \& Merge\\
SourceTree & 8 & Git/Hg Desktop Client\\
Sublime Text & 7 & Text Editor\\
\bottomrule
\end{tabularx}
\end{table}

\todo{Use the same terms from survey, not toolset fragmentation}
\todo{"Big bang up front"}
\subsubsection{\underline{Better Usability}}
The right toolset is essential for efficiently developing solution and resolving conflicts.
Usability is a major factor that determines whether a toolset will support practitioner's workflows instead of hindering them.
Based on the results of the survey, we found that practitioners rate usability as the most desired improvement for their current merge tools (I1).

Usability within a particular tool is important, but these usability concerns also stretch across multiple tools that serve similar use cases and must operate in sync with each other.
Our participants indicated that they use 2.50 merge tools on average.
With multiple tools being used during merging and conflict resolution, toolset fragmentation is a real concern for practitioners.
For instance, during the interviews, P1 demonstrated resolving a typical merge conflict using four different tools and said: 
\begin{displayquote}
\textit{``I have to jump around between tools and copy and paste version numbers from one to... See, this is why [toolset] integration matters.''}
\end{displayquote}

This frustration is understandable for developers whose workflows frequently get interrupted by tool switches. Psychology studies~\cite{Meiran2000}\cite{gopher2000switching} have found that task switching comes with costs in performance and mental fatigue, and, in 1992, Gerald Weinberg highlighted the problem of toolset fragmentation within engineering teams~\cite{Weinberg1992}. 
This motivated a survey question asking, \textit{``How often do you find that having multiple tools has been a problem in your development workflow?''} We received 121 responses with a mean score of 2.04 (\textit{Rarely} on a 5-point Likert scale from \textit{Never} to \textit{Always}). 

In comparing the effect of toolsets on problems in development workflow, we recognize that issues with particular tools could bias practitioner perceptions.
Therefore, we examined the tool lists of participants that chose \textit{Never} against the tool lists of participants that chose \textit{Rarely} and found that these groups shared 9 out of 10 of the most common tools.
Since both groups are responding about problems in the same set of tools, we conclude that both groups had different perceptions about the difficulties in using multiple tools as opposed to any particular combination of tools.

\subsubsection{\underline{Better Filtering of Less-Relevant Information}}
Version control systems (VCS) and bug tracking systems provide insufficient support for detailed analysis of software evolution and information retrieval~\cite{fischer2003release_history}.
For software practitioners using \texttt{git}, and other VCS, it is often easier to write scripts that accommodate their particular information needs by augmenting the capabilities of the VCS.
During the interviews, P1 described writing several scripts in order to locate particular historical commits that relate to a current merge conflict.
P9 also described a tool, \texttt{git-diff}, developed as part of their efforts to add additional difference analysis functionality across branches:
\begin{displayquote}
\textit{``git-diff will just do the diff based on the SHAs... we're adding metadata and cherry picking, so the SHAs are always going to be changing... this actually allows us to do a comparison based on SHA but then fall back to author and commit title and metadata information... It also hooks into GitHub labels and uses the labels on the project to do some more advanced heuristics.''}
\end{displayquote}

In our survey results, we see that practitioners rank \textit{better ways of filtering out less relevant information} (I2) as the second highest improvement needed in modern merge toolsets.
Combined with our interviews, we find that the expansion of metadata capabilities in GitHub, GitLab\footnote{https://gitlab.com/}, and other cloud-based repository hosting services are also driving the development of scripts and tools to utilize this data in practitioners' workflows.
Further work is needed to bring these capabilities from single-purpose scripts into common toolsets used by all practitioners.

\subsubsection{\underline{Better Exploration of Project History}}
Notably, many tool-related complaints in the interviews came from within the context of the \textit{Archeology Lens}~\cite{mihai_lenses}, meaning that developers find tool support lacking when trying to explore older history to understand the evolution of a project or even a single line of code. In the words of P1, 

\begin{displayquote}
\textit{``Give me a way to explore the history. To drill down, to go back up, you know? To resurface and understand what happened.''}
\end{displayquote}

\todo{need more explaining here}

\subsubsection{\underline{Information Presentation}}
Complaints about information presentation are a result of bad or inconsistent tool design. For instance, some complaints about inconsistent terminology, coloring, and organization across different resolution tools required developers to reorient themselves within each tool when switching contexts. While these are problems with the way these tools work together, individual tool usability was also questioned. For instance, one survey participant says, 
\begin{displayquote}
\textit{``Tools don't make it easy to work with two arbitrary revisions side by side.''}
\end{displayquote}

\todo{need more explaining here}

\begin{table}[!htbp]
\renewcommand{\arraystretch}{1.3}
\caption{How much software practitioners trust their merging, history exploration, and/or conflict resolution tools}
\label{survey_tool_trust}
\centering
\begin{tabularx}{0.45\textwidth}{@{}r|*{10}{C}c@{}}
\toprule
Trust Level & Response Count & Response \%\\
\midrule
Completely & 20 & 17\\
A lot & 50 & 41\\
A moderate amount & 41 & 34\\
A little & 10 & 8\\
Not at all & 0 & 0\\
\bottomrule
\end{tabularx}
\end{table}


\subsubsection{\underline{Tool Mistrust/Transparency}}
Tool mistrust seemed to come from being unsure about what the tool was really doing. Many merging tools obscure the steps that are actually being taken, making developers hesitant to trust that the appropriate steps have been initiated by the tool. 
\begin{displayquote}
\textit{``I've never trusted the merge tools, in a way. Or the diff tools. It would always just make me skittish. So my overall perception is that I'm scared of them. Sometimes I'll even manually go and do the merge myself rather than use a tool. Just because I've had several times where it's a bad merge, and I broke some code.''}
\end{displayquote}
This quote comes from P4, a software developer with 10 years of experience.
This prompted us to ask developers how much they trust their merging, history exploration, and/or conflict resolution tools. Of 121 respondents to this question, 91.74\% said that they trusted their tools at least \textit{A moderate amount}, and 57.86\% said they trust their tools \textit{A lot} or \textit{Completely} (Table \ref{survey_tool_trust}). This raises the questions: 
\begin{itemize}
\item How much tool trust is enough? 
\item What is the trust threshold for not using a tool anymore?
\end{itemize}
If we consider this problem conservatively, nearly 1 in 10 software practitioners are using tools that they cannot trust (see Table~\ref{survey_tool_trust}). Since no previous work exists relating to minimum acceptable levels of trust in a toolset, it is possible that up to 42.1\% of developers (those with \textit{A moderate amount} of trust or less) operate within this gap in tool trust.

As seen in Sections \ref{RQ1} and \ref{RQ2}, developers mentioned both size and complexity in the interviews, which raised questions about how developers view each of these metrics. More specifically, we wanted to know how well developers think tools support increased size and complexity.  
This motivated a survey question to compare increased size to increased complexity by asking participants to rate how well their tools handle the following kinds of conflicts:

\begin{itemize}
	\item Simple, small merge conflict resolutions
	\item Simple, large merge conflict resolutions 
	\item Complex, small merge conflict resolutions
	\item Complex, large merge conflict resolutions
\end{itemize}

Our results show that current tools handle increasing the size of the conflict better than increasing the complexity of the conflict. This trend was seen across all experience levels, as visualized in Figure \ref{size_vs_complexity}.

\begin{figure*}[!t]
\centering
\includegraphics[width=\textwidth]{ConflictComplexityVsSize.pdf}
\caption{Effectiveness of developer tools in supporting varying levels of size and complexity (by developer experience). Values are number of responses per answer, and bubble size indicates the number of responses for comparison purposes.}
\label{size_vs_complexity}
\end{figure*}

\subsubsection{\underline{Tool functional deficiencies}}
When told that our study was seeking to find difficulties in resolving merge conflicts, some participants prepared examples of functional deficiencies in conflict resolution tools. P3 shared a specific case from a backporting~\cite{gutzmann2009backporting} experience. In a large system, it is extremely hard to understand when code was deleted. The \texttt{--reverse} option for the \textit{Git Blame} tool offers this functionality, but as P3 points out:  

\begin{displayquote}
\textit{``It still requires thorough background knowledge of the code base. Also I work on large changes in time in hot paths. This could mean many additions and deletions that would lead to further confusion.''}
\end{displayquote}

 This illustrates the case of a tool that performs well in most cases but does not scale well to larger and more complex codebases.
 
 P1 mentions a specific issue that he encounters to illustrate that tools work with a shallow understanding of a merge conflict's process. He describes a conflict where one branch cherrypicks changes from the other, then the second branch reverts the same changes that were cherrypicked. When the two branches merge back together, tools have trouble supporting the resolution of the more complex conflict:
 \begin{displayquote}
 \textit{``So cherrypick on one branch, revert on the other, equals disaster. Most tools... just do three-way branch because they limit their vision to those three parts. They don't try to understand the actual scenario.''}
 \end{displayquote}
 Using this in light of our top difficulties from Table \ref{survey_res_diffs} suggests that tools would benefit from assisting developers in more complex situations by providing them with more specific information about what caused the conflict (in this case, the combination of cherrypicking and reverting).