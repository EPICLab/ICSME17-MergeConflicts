\subsection{\textbf{RQ1:} How do developers approach merge conflicts?}\label{RQ1}

We asked survey participants to rate how much each of 9 factors affected their perceptions of difficulty in approaching a merge conflict.
We received 162 responses using a 5-point Likert scale indicating the degree of effect on conflict difficulty (1 being \textit{Not at all}, 3 being \textit{A moderate amount}, and 5 being \textit{A great deal}).

Results of the survey are presented in Table~\ref{survey_merge_conflicts}.
We present and discuss in detail the top 4 factors based on mean score.
The factors are grouped into \textit{Artifact-based Factors} and \textit{Knowledge-based Factors}.


\subsubsection{\underline{Artifact-based Factors}}
Typical proxies used by lay practitioners for code complexity have included cyclomatic complexity~\cite{fenton2000quantitative}, atomicity~\cite{khelladi2016ad}, number of lines of code, number of files, etc.
However, we find that the practitioner's perceptions of complexity are just as important in determining how they approach the resolution of a merge conflict.
Two of the top 4 factors refer to perceptions about the complexity of merge conflicts (F1, F3), and only one of the top 4 factors indicates a specific metric of complexity (F4).

The perception of complexity can affect whether a practitioner resolves it immediately, when it is small, or whether they wait to examine the conflict when further resources can be allocated.
In the interview, P8 commented:
\begin{displayquote}
\textit{``Small is always easy. A 1-line merge conflict is always easier to resolve than a 400-line merge conflict.''}
\end{displayquote}

If a merge conflict is perceived to be large or complex, a practitioner may decide to forgo attempting to resolve it through code manipulation and choose to revert changes instead.
This ``nuclear option'' requires practitioners to disrupt the development flow, set aside their current development work, and potentially remove ``good'' code that was not part of the conflict in order to return to a non-conflicting state.

Further, when integrators are preparing code for production environments they prioritize merge conflicts for code review based upon the perceived difficulty of resolving the affected code.
We find that these decisions rely on humanistic factors as much as they rely on data-driven metrics.
Therefore, we need metrics that are human-aware and take into account the perceived difficulties of merge conflicts.

\todo{
Other researchers have shown that [ref], knowledge/domain experituse is important. THis hsows that knowing hwo is hte right person to assist with resolving a merge conflict, (integrator), is very importnat.
}

\todo{from Rafa: "Jump?" Maybe a rough transition}
\todo{We intentionally left complexity to participants because we did not want to bias a certain metric}
\todo{Tie in positive AND negative results: Takeaway is that we need a better metric for human understandability}
\underline{\textit{Code Complexity:}} Cyclomatic complexity \cite{mccabe1976complexity} is a well-established measure of code complexity in software. However, it has also been criticized as being based on poor theoretical foundations \cite{Shepperd1988}, and not all software developers have experience using it. For this study, we intentionally left the interpretation of the word ``complexity'' to the individual participants.
In the interview, P4 commented on how the size of the conflict impacts its difficulty, saying,

\begin{displayquote}
\textit{``The size is usually not the issue. Just the complexity of the code that happens to be with the conflicting state.''}	
\end{displayquote}

Since P4 did not explicitly state whether he meant complexity of the lines of code in the conflict or complexity of the files in the conflict (which accounts for a bigger-picture view of the code), we asked participants about both line complexity and file complexity. Our results show that both line complexity (mean of 3.52) and file complexity (mean of 3.23) rank above \textit{Number of conflicting lines of code} (mean of 3.14). This suggests that complexity impacts the apparent difficulty of the conflict more than size. However, it is also possible that conflict size is part of what defines conflict complexity, which would mean that size cannot be more impactful than complexity. 

\todo{Possible need for an expertise locator? Need more research to get there?}
\subsubsection{\underline{Knowledge-based Factors}}
Our findings show that expertise in the area of conflicting code is a top factor in determining a merge conflict's difficulty (see F2 in Table~\ref{survey_merge_conflicts}).
Knowing the impact of knowledge-based factors allows us to understand the difference that practitioners attribute to having more information. Ideally, every developer would know everything about the code base, but this is often unrealistic in larger projects. Instead, other practitioners with expertise help fill this information gap.

When attempting to resolve a conflict alone, developers working in unfamiliar code often feel uncomfortable making decisions during a merge due to lack of familiarity with the code or impacts related to the code change\cite{CostaSarma}. 
Situations such as these raise questions relating to which person is best suited to resolve the conflict without introducing a new bug or unintentionally changing code behavior. Studies such as Costa et al.~\cite{CostaSarma} hope to alleviate some risk by selecting the best practitioner to resolve conflicts. Our findings confirm the need for expertise-location tools and encourage further research into selection of knowledgeable practitioners for merge conflict resolution.

In our interview, P5 explained a situation where he applied this technique: 

\begin{displayquote}
	\textit{``A lot of what I work on is\textellipsis in my own little area, so when there is a merge conflict in that area, I will know what to do\textellipsis But if I'm\textellipsis in the [unfamiliar part of project], then I'll\textellipsis get someone else to\textellipsis resolve the merge conflict for me\textellipsis It's someone else's, and I don't want to screw it up.''}
\end{displayquote}

This participant found knowledge of the effected code to be important enough that he would find someone who is familiar with the code around the merge conflict rather than resolving the conflict himself. In the event of non-intuitive code, this option is a helpful approach to resolution because it does not require the developer to fully understand the code before resolving. By taking this approach, P5 acknowledges the time incentive to avoid understanding the whole context of the conflict, but he also states that \textit{"It's someone else's"}, highlighting the idea that merge conflicts are both social and technical challenges.

%The results of our survey indicate that \textit{complexity of conflicting lines of code} and \textit{knowledge/expertise in the area of conflicting code} are factors that practitioners find have a direct effect on the perceived difficulty of a merge conflict.
%Although we found consensus across all demographics, the experience level of respondents had an affect on the degree to which they responded affirmatively for these factors (see Table~\ref{survey_merge_conflicts}).
%
%Practitioners with 1-5 years and 6-10 years of experience were only nominally in agreement that merge conflict difficulty is affected by \textit{complexity of conflicting lines of code} (mean: 3.31 and mean: 3.37 respectively) and \textit{knowledge/expertise in the area of conflicting code} (mean: 3.21 and mean: 3.42 respectively).
%As practitioner experiences increases, the level of agreement increases on both of these factors, with practitioners indicating 26+ years experience having the highest consensus (mean: 4.06, std. deviation: 0.77). 
%
%Larger, more complex projects typically require that developers have more experience and a higher degree of knowledge about the systems used in the project.
%As the size of a software project increases, so does the complexity of the underlying code~\cite{banker1993software}\cite{curtis1979third}.
%Therefore, more experienced practitioners are likely to have a broader range of understanding about both \textit{complexity of conflicting lines of code} and \textit{knowledge/expertise in the area of conflicting code} and their affects on merge conflict difficulty.
%
%\begin{figure}[!t]
%\centering
%\includegraphics[width=3.4in]{MeanProjectSize.pdf}
%\caption{Mean Likert Score by Project Size for \textit{knowledge/expertise in the area of conflicting code} category of factors that effect the perceived difficulty of merge conflicts.}
%\label{mean_likert_project_size}
%\end{figure}
%
%When analyzing by the size of project that practitioners normally work on, we found that \textit{knowledge/expertise in the area of conflicting code} was increasingly viewed as having an effect on merge conflict difficult as the practitioners' project size increased (see Figure~\ref{mean_likert_project_size}).
%The practitioners that usually work in project sizes of \textit{1 developer}, \textit{2-5 developers}, and \textit{6-10 developers} had relatively neutral responses regarding the effects of \textit{knowledge/expertise in the area of conflicting code} (mean: 3.12, mean: 3.38, and mean: 3.47 respectively).
%However, practitioners that usually work in project sizes of \textit{11-50 developers} and \textit{51+ developers} more positively indicated that \textit{knowledge/expertise in the area of conflicting code} has an effect on the difficulty of merge conflicts (mean: 4.00 and mean: 3.69 respectively).
%
%We also found that practitioners agree that \textit{non-functional changes (whitespace, renaming, etc.)} (mean: 2.16, std. deviation: 1.03) has no effect on the perceived difficulty of a merge conflict.
%Non-functional changes are textual conflicts that can be resolved without consideration of dependencies, performance, or scope.
%Practitioners have less to reason about when resolving merge conflicts that involve only textual conflicts, and therefore are unlikely to perceive \textit{non-functional changes} as affecting the difficulty of merge conflicts.

%Intuitively, we would assume that an increased perception of \textit{knowledge/expertise in the area of conflicting code} as difficulty factor would correlate with a decreased perception of \textit{complexity of conflicting lines of code} as a difficulty factor of merge conflicts.
%This relationship would indicate that a developer's expertise allows them to rationalize about merge conflicts with higher complexity, since they have a higher degree of past knowledge in that area of conflict.
%However, we find a statistical significant correlation between \textit{complexity of conflicting lines of code} and \textit{knowledge/expertise in the area of conflicting code} (Pearson's correlation coefficient: 0.2766, p-value: 0.0003664, 95\% confidence interval: 0.1279 0.4132).
%This suggests that ...

%%%%%%%%%%%%%%%%%%%%%%%%%%%%%%%%%%%%%%%%%%%%%%%%%%%%%%%%%%%%%%%%%

%\subsubsection{Interviews}
%%The interview results suggest that developers approach merge conflicts...
%
%\subsubsection{Survey}
%Our survey suggests that regardless of gender, developer role, experience level, project size, and source distribution model, software practitioners say that the following factors affect the difficulty of a merge conflict most: 
%\begin{itemize}
%\item \textit{Complexity of conflicting lines of code}
%\item \textit{Your knowledge/expertise in area of conflicting code}
%\end{itemize}
%
%Similarly, software practitioners across every measured demographic perceived the following factors to be less important when deciding the difficulty of a merge conflict:
%\begin{itemize}
%\item \textit{Non-functional changes (whitespace, renaming, etc)}
%\item \textit{Number of files in the conflict}
%\end{itemize}
%
%While survey participants did not agree that non-functional changes strongly factor into the difficulty of a merge conflict, it is still worth noting that several interview participants named non-functional changes, such as large refactor or reformatting changes, as a cause for merge conflicts. This suggests that non-functional changes may increase the likelihood of a merge conflict happening, but they do not contribute to the conflict's difficulty.
%
%However, some demographics do view certain difficulties. For instance, open-source developers think that \textit{Atomicity of change sets in the conflict} impacts the difficulty, while closed-source developers and people who split their time evenly think that atomic change sets have no effect on the difficulty. This may be explained by the findings in Rigby et al\cite{OSS_smaller_commits}, which shows that open-source projects tend to review smaller changes than closed-source projects because "The small size lets reviewers focus on the entire change, and the incrementality reduces reviewers’ preparation time and lets them maintain an overall picture of how the change fits into the system." It is possible that our result reflects this difference of culture.
%
%We also found that Project Maintainers say that \textit{Time to resolve a conflict} has an effect, while no other role agrees. This suggests that those in a maintainer role may be more subject to time-related constraints such as maintenance or release schedules. 
%
%\comment{Project Managers say no effect because they focus on project schedules, not conflict resolutions, i.e. they are higher level/abstraction?}
%
%\todo{might be previous work}
%Support and infrastructure roles (System Engineer, Sys Admin, System Architect, DevOps) emphasized that \textit{Dependencies of conflicting code on other components} have more of an effect than other roles did. This might be due to infrastructure systems being maintained in a live environment, or systems that are currently in production use and at risk of real-time dependency failures. 
%
%Developers on projects of size 1 say that \textit{Dependencies of conflicting code on other components}. Because no other project sizes agree with this idea, we hypothesize that this could be due to their high dependence on external code because of the software production limitations of a 1-developer team.
%
%We also found that the group of developers with 21-25 years of experience frequently contradicted general consensus, but it seems more likely that these differences were simply due to the group's small sample size (4).

%We asked participants how much they trust their merging, history exploration, and/or conflict resolution tools, and 57.9\% of participants reported that they trusted these tools either \textit{A Lot} or \textit{Completely}. While this is a majority of developers, this still leaves a significant number of people (42.1\%) who trust their tools \textit{A moderate amount} or \textit{A little}. Though we had the option for \textit{Not at all}, no participants selected this option, presumably because users stop using tools that they do not trust at all. While we found no previous work discussing the threshold for how much users must trust tools for a good tool experience, we postulate that users who cannot trust their tools \textit{A Lot} or \textit{Completely} will avoid relying on such tools too much.