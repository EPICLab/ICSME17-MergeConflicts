\begin{abstract}
\emph{Background:}
Merge conflicts occur when software projects distribute development work among multiple software practitioners.
Tool builders and researchers have focused on the prevention and resolution of merge conflicts, but there is little empirical knowledge about the how practitioners rationalize the merge conflict resolution process.

\emph{Aims:}
The objective of this study is to empirically investigate software practitioner perceptions in the approach, resolution, and tools used for managing merge conflicts.

\emph{Method:}
We conducted semi-structured interviews of 10 software practitioners across 7 organizations that produce both open- and closed-source software.
Using qualitative card sorting techniques, we determined key concepts and perceptions that are extended by conducting a survey of 162 practitioners.

\emph{Results:}
Practitioners are directly impacted by the perceived complexity of conflicting code, and alter both the timeline to resolve and the methods employed for conflict resolution based upon that initial perception.
Information about conflicting code, including the history of related changes and availability of experts in that particular area of code, are crucial unmet needs for practitioners.
We find that merge tools have been effective at addressing smaller conflicts, but do not scale in terms of usability, information presentation, or history exploration when large/complex merge conflicts arise.

\emph{Conclusions:}
Practitioners' perceptions alter the impact of tools and processes that have been designed to preemptively and efficiently resolve merge conflicts.
Understanding whether practitioners will react according to standard use cases is important when creating human-oriented tools to support development processes.
Our results contribute to relevant impacts on research and practices, and to topics that deserve further study.

\end{abstract}