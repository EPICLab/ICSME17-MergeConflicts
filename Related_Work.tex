\section{Related Work}\label{related_work}

\todo{How Do Software Engineers Understand Code Changes? - An Exploratory Study in Industry}
Gousios et al. \cite{integrator_perspective} perform a similar study in which they ask integrators to describe difficulties in maintaining their projects and code contributions. 
Both their paper and ours seek to identify a set of difficulties and find that there are many sides of the problem. They showed that integrators also have problems with their tools, have trouble with non-atomic changesets, and even rank \textit{git knowledge} reasonably high on the list of biggest challenges (top 30\% of factors). Specifically, R42 in their study states that \textit{``most [contributors] don’t know
how to resolve a merge conflict''}. They also experience what our study refers to as \textit{Project Culture}, but integrators tend to focus on tasks such as creating a project's culture by coordinating development efforts, tactful pull request rejections, and maintaining the vision of the project in spite of differing opinions. While both papers explore perceived difficulties, our work seeks to understand the challenges of surrounding merge conflicts; whereas, Gousios et al. found challenges associated with being an integrator, which focus more on social aspects of the pull-based development.

De Rosso et al \cite{DeRosso2016} used StackOverflow to create a list of ``misfits'' to indicate some problems with the Git workflow. Later, they create a tool, Gitless, to alleviate some of the pain involved in using Git. In their evaluation of the tool with 11 participants, participant success rates for unnecessarily complex tasks like  switching branches while resolving a merge conflict were 36\% higher (up to 91\%) for using Gitless than when using Git. While this is still in early stages and is promising for future work, it is a lab study with threats to generalizability and needs further evaluation. 
 When asked about satisfaction, efficiency, difficulty, and frustration using Gitless, users generally preferred Gitless to Git. However, their expert Git user preferred Git in all of these categories. Our experienced interview participants also discussed complex situations in which they wished for more powerful tools, so the Gitless approach of simplifying the main workflow may only be ideal for newer Git users. It is possible that more experienced users will not benefit from these improvements because of the need to handle less-usual use cases. This suggests that as a whole, the tool needs implementation that also focuses on more advanced needs. However, it does show that our study is not the only one that recognizes a gap in practitioner needs and tool implementation.

Similar to our paper, Codoban et al. seeks to evaluate developer understanding and usage of code history \cite{mihai_lenses}. Our results show that history exploration factors into the difficulty of a merge conflict a moderate amount (mean of 3.03 on a 5-point likert scale). Because of this, their findings apply to a subset of the merge conflict resolution process, so some of their conclusions such as tool limitations in usability and history visualization may contribute to our independent findings that some participants want better usability (mean of 3.43) and better graphical presentation of information (mean of 3.14) from their merge conflict resolution tools.