\section{Related Work}\label{related_work}

Gousios et al. \cite{integrator_perspective} conduct a study in which they ask integrators to describe difficulties in maintaining their projects and code contributions. 
They showed that integrators have problems with their tools, have trouble with non-atomic changesets, and rank \textit{git knowledge} in the top 30\% of their list of biggest challenges. 
Gousios et al. additionally conducted a study into the challenges of the pull-based model from the perspective of contributors~\cite{gousios2016work}. 
They found that most challenges relate to code contribution, the tools and model used to contribute, and the social aspects of contributing (specifically highlighting merge conflicts).
These works focus on the collaborative processes that goes into contributing to open-source projects and operating as integrators within them, whereas we examine the issues inherent to merge conflicts and the tools built to support their resolution.

Guzzi et al.~\cite{Guzzi2015} conducted an exploratory investigation and tool evaluation for supporting collaboration in teamwork within the IDE.
They found that developers working within a variety of companies were able to quickly and easily resolve merge conflicts, and to do so using merge tools.
However, they also note that although automatic merging was used, their participants also manually checked each conflict and suggest that this reveals some mistrust of tools.
Guzzi et al. further explain that their interviewees avoid merge conflicts by using strict policies and software modularity.
Their results compliment our findings that toolset mistrust is a major concern, and that policies need to be implemented in order to avoid complex merge conflicts.

Codoban et al.~\cite{Mihai_lenses} seek to evaluate developer understanding and usage of code history. Our results show that tool support during history exploration factors into the difficulty of a merge conflict a moderate amount (N10). This result independently verifies their findings that practitioners experience tool limitations in usability (I1) and history visualization (I4).