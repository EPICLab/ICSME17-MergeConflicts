\footnotetext{\IEEEauthorrefmark{1} First and second author contributed equally to this work.}

\todo{Remove paragraph orphans prior to submission.}
\todo{Add survey questions to companion website prior to submission. Anonymize for ICSME submission.}

\section{Introduction}\label{introduction}

Collaborative development is essential for the success of large projects~\cite{hattori2010syde}, and is enabled by version control systems. 
In Git and other version control systems, developers work on their changes in seclusion, and periodically synchronize them by merging with the main line of development. 
Although a large number of commits cleanly merge, parallel changes can overlap, leading to merge conflicts. Kasi et al.~\cite{cassandra} and Brun et al.~\cite{Brun2011}, in their studies of several open source projects, found that merge conflicts occur in approximately 19\% of all merges.

Resolving merge conflicts is nontrivial, especially when changes diverge significantly, making their synchronization difficult. 
The resolution process can be tedious and can cause delays as developers figure out how to approach and resolve conflicts~\cite{cassandra}. 
Poorly-performed merge conflict resolutions have been known to cause integration errors~\cite{bird-branches-conflict}, workflow disruptions, and jeopardize project efficiency and timelines~\cite{estler2014awareness}. 

Developers are aware of the merge-resolution ``pains" and follow different informal processes to avoid having to resolve conflicts (e.g., sending out emails to the rest of the team, performing partial commits, or racing to finish changes) \cite{deSouza2003breaking}\cite{cataldo2008distributed_dev}.
Unfortunately, this workflow causes changes to diverge more and makes merge conflict resolution harder~\cite{Brun2011}. 


Past work has examined different mechanisms for proactive conflict detection~\cite{Brun2011}\cite{palantir}\cite{Guimaraes}, proposed tools for resolving merge conflicts~\cite{nishimura}\cite{mens2002state}, and discussed advantages of syntax- and semantic-aware merges \cite{danny_refactorings}\cite{hunt2002extensible}. 
However, merge conflicts are a regular occurrence in collaborative development, and at some point during parallel development, practitioners will face and have to resolve merge conflicts.

Our goal in this work is to understand how software practitioners perceive the difficulty of merge resolutions, how they approach a merge conflict, and how well current tools support their conflict resolution needs. Without such knowledge, tool builders might be building on wrong assumptions and researchers might miss opportunities for improving the state of the art.
%Despite the focus on reducing the occurrence of merge conflicts, prior has thus far not talked to practitioners to understand their perceptions about merge conflicts and how current tools support them.
Talking directly to practitioners is crucial to understanding problems in the context of the work~\cite{fritz2010using, sillito2006questions, de2008answering, ko2007information}.
%To fill this gap in knowledge, we performed an empirical study of software practitioners to understand their perspectives and experiences with merge conflicts in software development.

We interviewed 10 software practitioners about their experiences and perceptions of merge conflicts in the software development process. These interviews helped us understand how practitioners approach merge conflicts and their unmet needs.
To triangulate our findings and provide a broader understanding of practitioners' perceptions of merge conflicts and their difficulty, we deployed a survey to a larger population of software practitioners.
The survey sampled 162 participants, spanning both open source and commercial projects. 74.2\% of our participants had 6 or more years of software development experience, and reported that they face merge conflicts a few times a week.
%most frequently report encountering merge conflicts a few times a week. Participants spanned both open source and commercial projects.
%The interviews provide the basis for the survey, as well giving us personal accounts of merge conflicts.
%The survey provides a broader understanding of how perceptions permeate the larger software development community.

To understand the effects and implications of software practitioner perceptions, we answer the following research questions:

\begin{itemize}
\item \textbf{RQ1:} How do software practitioners approach merge conflicts?
\item \textbf{RQ2:} What unmet needs impact the difficulty of a merge conflict resolution?
\item \textbf{RQ3:} How well do tools meet practitioner needs for merge conflicts?
\end{itemize}

Based on the results of our interviews and survey, we found several factors that contribute to practitioners' perceptions of merge conflict difficulty.

We found that practitioners, when initially assessing a merge conflict, focus on the \textit{code complexity of the conflicting lines} and \textit{their own knowledge in the area of the conflict} as the top two factors when estimating the difficulty of a conflict. These concerns cause practitioners to alter their resolution strategy, and in some cases delay resolution.

After coming to an understand about the merge conflict, practitioners must resolve the conflict in order to return to normal development.
We found that, the key challenges that practitioners face when resolving conflicts is \textit{understanding the conflicting code}, \textit{their knowledge in the area of code conflict}, and having enough meta information about the conflicting code (who made the change, why, etc).

We also found that developer tools struggle to support merge conflicts as they increase in perceived size and complexity.
This could be partially alleviated by focusing on the tool improvements most desired by practitioners: \textit{better usability}, \textit{better information filtering}, and \textit{better history exploration}.

This paper makes the following contributions:
\begin{itemize}
\item We conduct exploratory semi-structured interviews with 10 software practitioners, then confirm these findings with a survey of 162 practitioners from around the world.
\item We provide empirically justified rankings of merge conflict difficulties based on practitioners' perceptions.
\item We expose disparities between practitioners' merge conflict support needs and development toolsets.
%\item We present an analysis of the different difficulties in merge conflict resolutions for sub-populations of software practitioners.
\end{itemize}