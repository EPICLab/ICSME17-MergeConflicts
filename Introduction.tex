\section{Introduction}\label{introduction}

\comment{\textbf{***distributed development process***}}

\comment{\textbf{***Conflicts occur often and cause integration errors****}}
\todo{Do we want to cite Caius/Iftekhar paper?}
Version control systems like Git offer easier collaboration via an automated merging process. This process enables large-scale distributed development, but as sophisticated as these tools are, they still require manual merging in ambiguous cases called merge conflicts. Sarma et al \cite{cassandra} shows that merge conflicts are as frequent as 19\% of merges, and Brindescu \& Ahmed et al (maybe) shows that [\% of merges that are conflicts] of merges result in conflict. Resolving merge conflicts can be a costly process that delays projects while developers decide how to approach and resolve the conflict \cite{cassandra}, and poorly-performed merge conflict resolutions frequently cause integration errors \cite{bird-branches-conflict}.

\comment{***current work on proactive detection***}

Several studies have examined models for proactive merge detection and prevention~\cite{Brun2011}\cite{palantir}\cite{Guimaraes}, proposed tools and systems for efficiently avoiding or resolving merge conflicts~\cite{nishimura}\cite{mens2002state}, or discussed advantages of syntax- and semantic-aware merges \cite{danny_refactorings}\cite{hunt2002extensible}. However, at some point in a software projects' development history, there is the chance that a merge conflict will still occur. 

\comment{***we need to know more about conflicts because...***}

Despite the focus in this area, few studies have actually talked to developers to understand their perceptions about merge conflicts. Talking directly to developers is crucial to understanding their problems from their perspectives, since some details of merge conflict resolution cannot be observed by mining a GitHub repository. Other studies have also recognized the importance of directly talking to developers about other problems such as understanding integrator difficulties \cite{integrator_perspective}.

Our research questions are as follows:
\begin{itemize}
\item\textbf{RQ1:} How do developers approach merge conflicts?\\
\item\textbf{RQ2:} What factors impact the difficulty of a conflict resolution?\\
\item\textbf{RQ3:} Do developer tools meet developer needs for merge conflicts?\\
\end{itemize}

In this study, we investigate these research questions with an interview of 10 developers and a survey of 226 software practitioners. We provide personal accounts from interviews of experiences in resolving merge conflicts and ranked lists of: (a) the 9 factors that developers use to assess the difficulty of a conflict, (b) the 10 factors that impact developer difficulty while resolving merge conflicts, and (c) 6 tool needs for merge conflict resolution.

When initially assessing the difficulty of a merge conflict, we found developers focus on the \textit{Complexity of conflicting lines of code} and \textit{Your knowledge/expertise in the area of conflicting code} as the top factors that impact their estimated difficulty, while non-functional changes like whitespace and renaming have little effect on this assessment. Later, during the resolution process, we found that developers confirmed \textit{How easy it is to understand the code involved in the merge conflict}, \textit{Your expertise in the area of code with the merge conflict}, and \textit{The amount of information you have about the conflicting code} as 3 important factors that impact the difficulty of the resolution.
