\section{Introduction}\label{introduction}

\comment{\textbf{***distributed development process***}}

\comment{\textbf{***Conflicts occur often and cause integration errors****}}

Version control systems like Git offer easier collaboration via an automated merging process. This process enables large-scale distributed development, but as sophisticated as these tools are, they still require manual merging in ambiguous cases called merge conflicts. Sarma et al \cite{cassandra} shows that merge conflicts are as frequent as 19\% of merges, and Brindescu \& Ahmed et al (maybe) shows that [\% of merges that are conflicts] of merges result in conflict. Resolving merge conflicts can be a costly process that delays projects while developers decide how to approach and resolve the conflict \cite{cassandra}, and poorly-performed merge conflict resolutions frequently cause integration errors \cite{bird-branches-conflict}. 

\comment{\textbf{***current work on proactive detection***}}
Several studies have examined models for proactive merge detection and prevention~\cite{Brun2011}\cite{palantir}\cite{Guimaraes}, and proposed tools and systems for efficiently avoiding or resolving merge conflicts~\cite{nishimura}\cite{mens2002state}, but at some point in any software projects' development history is the chance that a merge conflict will still occur.

\comment{\textbf{***we need to know more about conflicts because...***}}

However, few have actually talked to developers to understand their perceptions about merge conflicts. Talking directly to developers is important for understanding their problems from their perspectives.

\textbf{***This going to make the paper or break it ? refer to Danny/Mihai paper (lenses)***}

Our research questions are as follows:
\begin{itemize}
\item\textbf{RQ1:} How do developers approach merge conflicts?\\
\item\textbf{RQ2:} What factors impact the difficulty of a conflict resolution?\\
\item\textbf{RQ3:} Do developer tools meet developer needs for merge conflicts?\\
\end{itemize}

In this study, we investigate these research questions with an interview of 10 developers and a survey of 226 software practitioners.