\section{Introduction}\label{introduction}

\todo{Remove page numbers prior to submission. See note at the top of main.tex file.}
\todo{Remove paragraph orphans prior to submission.}
\todo{Add survey questions to companion website prior to submission.}
\todo{Per Anita's comments, the following prompts needs to be used to rewrite the Introduction section:\\1. What is the problem?\\2. Why is it a problem?\\3. What has been done so far?\\4. What have you done?}

Team collaboration is essential for the success of multi-developer projects~\cite{hattori2010syde}.
Version Control Systems (VCS), such as Git, are typically used to to enable both collaboration and large-scale distributed development.
The distributed nature of these systems requires synchronization between repositories.
For Git and other distributed VCS, synchronization occurs through merge commits that add changes on top of the currently tracked versions of files.

Most commits are cleanly merged into a repository.
However, Kasi et al.~\cite{cassandra} found that merge conflicts occur in approximately 19\% of all merges.
Resolving merge conflicts can be a costly process that delays projects while developers decide how to approach and resolve the conflict \cite{cassandra}, and poorly-performed merge conflict resolutions frequently cause integration errors \cite{bird-branches-conflict}.
Software practitioners responding to merge conflicts have to consider the origins of the conflict, the dependencies associated with the conflicting code, and the possible solutions for resolving the discrepancies between conflicting versions of files.
The difficulties associated with these aspects are non-trivial and require attention and awareness from practitioners in order to successfully mitigate associated concerns.

Several studies have examined models for proactive merge detection and prevention~\cite{Brun2011}\cite{palantir}\cite{Guimaraes}, proposed tools and systems for efficiently avoiding or resolving merge conflicts~\cite{nishimura}\cite{mens2002state}, and discussed advantages of syntax- and semantic-aware merges \cite{danny_refactorings}\cite{hunt2002extensible}. However, at some point in a software projects' development history, it is given that a merge conflict will occur. 

Despite the focus on reducing the occurrence of merge conflicts, no prior studies have talked to developers to understand their perceptions about merge conflicts.
Talking directly to developers is crucial to understanding their problems in context~\cite{fritz2010using, sillito2006questions, de2008answering, ko2007information}.
Gousios et al.~\cite{integrator_perspective} talked directly to integrators through interviews and surveys in order to understand the perspectives of integrators (an open-source project's core members, who are responsible for integrating contributions into the project).
We conduct similar studies on software practitioners in order to understand their perspectives and experiences with merge conflicts in software development.

This paper examines merge conflicts from the developer standpoint, seeking to understand the perceived difficulties and the processes of resolving them.
We then identify target areas for improvement in merge conflict resolution tools.
To investigate these factors, we interview 10 software practitioners about their experiences and perceptions of merge conflicts in the software development process.
To triangulate our findings, we deployed a survey.
The survey samples 162 participants, of which 95\% have held the role of Software Engineer/Developer, and 74.2\% have 6 or more years of software development experience.
The interviews provide the basis for the survey, and personal accounts of merge conflicts, and the survey provides a broader understanding of the extend to which those perceptions permeate the larger software development community.
Analyzing the data from both, we answer three research questions:

\todo{Interleave motivations for each RQ rather than a full block of RQs}
\vspace{3px}
\noindent\textbf{RQ1:} \textit{How do developers approach merge conflicts?}\\
\noindent\textbf{RQ2:} \textit{What unmet needs impact the difficulty of a merge conflict resolution?}\\
\noindent\textbf{RQ3:} \textit{Do developer tools meet developer needs for merge conflicts?}
\vspace{3px}

Based on the results of our study, we found several factors that contribute to the overall perception of difficulty when working with merge conflicts.

We found that developers, when initially assessing a merge conflict, focus on the code complexity of conflicting lines and their knowledge in the area of the conflict as the top factors that impact their estimated difficulty.
Non-functional changes like whitespace and renaming have little effect on this assessment. 

We found that, during the resolution process, developers identified code understandability, their knowledge in the area of code conflict, and the available information about the conflicting code as key factors that impact the difficulty of resolving merge conflicts.

We also found usability and information retrieval functionality as being the most desired improvements for developer tools to support the evaluation and resolution of merge conflicts.

This paper makes the following contributions:
\begin{itemize}
\item We conduct exploratory semi-structured interviews with 10 software practitioners, then confirm these findings with a survey of 162 practitioners from around the world.
\item We provide empirically justified rankings of merge conflict difficulties based on practitioners' perceptions.
\item We expose disparities between practitioners' merge conflict support needs and development toolsets.
\item We present an analysis of the different difficulties in merge conflict resolutions for sub-populations of software practitioners.
\end{itemize}