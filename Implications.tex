\section{Implications}\label{implications}
This paper examines the ways that software practitioners perceive the merge conflict resolution process.
We discuss the implications, from their perspective, and break down the implications for researchers, tool builders, and managers.

\subsection{For Researchers}
Our results inform future research in the area of merge conflicts from a human perspective.
Our findings from Sections~\ref{artifact-based-factors} and \ref{knowledge-based-factors} indicate that practitioners' knowledge gaps during merge conflict resolution are of similar importance as code-centric metrics of code complexity and SLOC.

\textit{Complexity of conflicting lines of code} (F1) and \textit{Your knowledge/expertise in the area of conflicting code} (F2) represent our top two factors in the assessment of difficulty when approaching a merge conflict.
These factors illustrate the divide between common metrics and the information needs of practitioners performing the resolution.
 
Our findings confirm that informativeness of commit messages (N8) has a moderate impact on practitioners' perceptions, but with seven other factors ranked higher, it receives more research focus than might otherwise be warranted.
Future research should also focus on knowledge- and comprehension-based difficulties of  merge conflict resolutions.

We also highlight nine unmet needs and nine factors of difficulties which have no statistical differences across experience, project size, open/closed source, or role.
This population-wide consensus indicates that practitioners unanimously want these unmet needs resolved, and these factors lessened.

\subsection{For Tool Builders}
Tools are meant to solve problems, but our results show that addressing the human needs of  practitioners in the areas of context-awareness and information retrieval during merge conflict resolutions is also important.
Mainstream tools must help find expertise in areas of conflicting code.
This result agrees with and adds to work done by Costa et al.~\cite{CostaSarma}.
Additionally, tool builders need to provide more ways for practitioners to gather information about the change impact of merge conflicts, including rationales for those changes when possible.

Version control systems provide a easy way of storing and retrieving recent history, but examining older version history at scale is still an open concern among practitioners.
Tool builders should work to address this unmet need by leveraging research into search systems for developer-assistance~\cite{nabi2016putting} and machine learning-based code assistances~\cite{bradley2011history_exploration} to provide practitioners with more expressive tools for history exploration.

Tools specific to integrators, who merge more often and more frequently than regular practitioners, are lacking in depth of features specific to this group.
Support for more advanced and more specific use cases is critical to supporting those who deal with these large, complex merges. 

\subsection{For Managers}
Practitioners find the merge conflict resolution process to be challenging due to a lack of information, lack of expertise, and a lack of code comprehensibility (N1-N3). 
To reduce these challenges, managers should both encourage and allot time for code reviews and paired conflict resolution sessions. 
Allowing better communication between different stakeholders allows domain experts to express concerns about impacts of changes, allows authors to understand the intentions of others' code changes, and allows developers to put the collective knowledge of the group to the task of finding the best resolution for the conflict.

Further work is needed for managers to effectively determine the most appropriate pairings of experienced practitioners and newcomers in order to resolve merge conflicts and prevent future issues.
Expertise recommendation tools such as TIPMerge~\cite{CostaSarma}, Ensemble~\cite{xiang2008ensemble}, or techniques such as the H-Index~\cite{bornmann2005does} should be further examined for applicability in industrial development settings.