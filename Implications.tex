\section{Implications}\label{implications}
\todo{
The needs of developers (OSS/Commercial) are the same, they don't exist differently, and yet they are unmet needs...
We examined demographics, they had the same challenges, needs, tool issues.
Implications in 3 sections as previously written...
}

This paper provides a look at the way that software developers perceive the merge conflict resolution process and highlights the most important parts, from their perspectives. We break down the separate implications for researchers, tool builders, and managers as follows:

\todo{These... these suck. Redo in light of Anit a discussion. (Commit messages/}
\subsection{For Researchers}
Our results inform future research in the area of merge conflict difficulties.
We provide information about which measurable factors help software practitioners' initial assessment of the difficulty of a merge conflict (i.e. \textit{Complexity of conflicting lines of code}, \textit{Number of conflicting lines of code}, \textit{Time to resolve a conflict}). We also found several factors that impact the difficulty of the merge conflict resolution which can be investigated lThis information allows researchers to compare these perceived difficulties to difficulties in practice and better understand where to focus research efforts.
\todo{
Program comprehension is a major thing for merge conflicts, the focus on maintenance and bug-fixes have primarily been the focus of the ICPC conference. Perhaps more focus needs to be placed on code comprehension in regards to merge conflicts. 
}
 
The intention of this paper is to provide an investigative platform for future work into how developers perceive difficulties related to merge conflict resolution. Notably, we have already seen work in the area of history exploration \cite{mihai_lenses}.

\todo{Give them some concrete things to work on}
\subsection{For Tool Builders}
From the interviews, this study provides direct quotes from developers relating to trust in tools, usability, presentation of information, and functional deficiencies. From the survey, we provide tool builders with a list of priorities for tool improvements, as suggested by 10 interviewees and ranked by 226 software developers. With this information, tool builders can make more informed decisions about how to allocate resources.

\todo{Managers need to facilitate better paired resolutions/code reviews, code understanding is important, so pairing experienced devs with new devs might be helpful/expertise differences}
\todo{Ripple effect paper http://ieeexplore.ieee.org/document/5710950/}
\subsection{For Managers}
As Table \ref{survey_roles} shows, 53 of the 54 project managers in our survey also identified as software developers themselves. As a result, many project managers likely have their own experiences and preferences of what is difficult and what their own tool needs are. In this paper, we provide a reference for managers to understand a more general set of difficulties and needs for a larger group of software developers. While each team's needs will vary, our results give managers another tool with which to evaluate potential tools and management practices.