\section{Implications}\label{implications}
\subsection{For Researchers}
Our results inform future research in the area of merge conflict difficulties.
We provide information about which measurable factors help software practitioners' initial assessment of the difficulty of a merge conflict (i.e. \textit{Complexity of conflicting lines of code}, \textit{Number of conflicting lines of code}, \textit{Time to resolve a conflict}). We also found several factors that impact the difficulty of the merge conflict resolution which can be investigated lThis information allows researchers to compare these perceived difficulties to difficulties in practice and better understand where to focus research efforts. We provide a somewhat shallow description of these difficulties because participants themselves have trouble identifying the specific difficulties that merge conflicts give them. 
The intention of this paper is to provide an investigative platform for future work into how developers perceive difficulties related to merge conflict resolution. Notably, we have already seen work in the area of history exploration \cite{mihai_lenses}.
\subsection{For Tool Builders}
From the interviews, this study provides direct quotes from developers relating to trust in tools, usability, presentation of information, and functional deficiencies. From the survey, we provide tool builders with a list of priorities for tool improvements, as suggested by 10 interviewees and ranked by 226 software developers. 
\subsection{For Managers}
As Table \ref{survey_roles} shows, 53 of the 54 project managers in our survey also identified as software developers themselves. As a result, many project managers likely have their own experiences and preferences of what is difficult and what their own tool needs are. In this paper, we provide a reference for managers to understand a more general set of difficulties and needs for a larger group of software developers. While each team's needs will vary, our results give managers another tool with which to evaluate potential tools and management practices.