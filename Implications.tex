\section{Implications}\label{implications}
This paper examines the ways that software practitioners perceive the merge conflict resolution process.
We discuss the implications, from their perspective, and break down the implications for researchers, tool builders, and managers.

\subsection{For Researchers}
Our results inform future research in the area of merge conflicts from a human perspective.
Our findings from Sections~\ref{artifact-based-factors} and \ref{knowledge-based-factors} indicate that practitioners' knowledge gaps during merge conflict resolution are of similar importance as code-centric metrics of code complexity and SLOC.

\textit{Complexity of conflicting lines of code} (F1) and \textit{Your knowledge/expertise in the area of conflicting code} (F2) represent our top two factors in the assessment of difficulty when approaching a merge conflict.
These factors illustrate the divide between common metrics and the information needs of practitioners performing the resolution.
 
Our findings confirm that informativeness of commit messages (N8) has a moderate impact on practitioners' perceptions, but with seven other factors ranked higher, it receives more research focus than might otherwise be warranted.
Future research should also focus on knowledge- and comprehension-based difficulties of  merge conflict resolutions.

We also highlight nine unmet needs and nine factors of difficulties which have no statistical differences across experience, project size, open/closed source, or role.
This population-wide consensus indicates that practitioners unanimously want these unmet needs resolved, and these factors lessened.

\todo{
Program comprehension is a major thing for merge conflicts, the focus on maintenance and bug-fixes have primarily been the focus of the ICPC conference. Perhaps more focus needs to be placed on code comprehension in regards to merge conflicts. 
}
 
%The intention of this paper is to provide an investigative platform for future work into how developers perceive difficulties related to merge conflict resolution. Notably, we have already seen work in the area of history exploration \cite{mihai_lenses}.

\todo{Give them some concrete things to work on}
\subsection{For Tool Builders}
Tools should solve the targeted problem, but our results also suggest that they should enable developers by addressing developer needs for more context and analysis during a conflict resolution. Mainstream tools should help find expertise in areas of conflicting code, as Costa et al. did with TIPMerge tool~\cite{CostaSarma}. Additionally, they should provide a report of the impact of the changes involved in the conflict and provide relevant details about the circumstances surrounding the conflict.

Tool builders should address history exploration at scale, specifically for those in integrator roles. Support for more advanced and more specific use cases is critical to supporting those who deal with larger and more frequent merge-related tasks.
\todo{list RQ3 impls} 

\todo{Managers need to facilitate better paired resolutions/code reviews, code understanding is important, so pairing experienced devs with new devs might be helpful/expertise differences}
\todo{Ripple effect paper http://ieeexplore.ieee.org/document/5710950/}
\subsection{For Managers}
Practitioners who solve merge conflicts find the merge conflict resolution process to be challenging due to a lack of information, lack of expertise, and a lack of code comprehensibility (N1-N3). This suggests that managers should encourage code reviews and paired conflict resolutions. Allowing better communication between different stakeholders allows domain experts to express concerns about impacts of changes, allows authors to understand the intentions of others' code changes, and allows developers to put the collective knowledge of the group to the task of finding the best resolution for the conflict.