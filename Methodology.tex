\section{Methodology}\label{methodology}
\textbf{***We did a 2-step approach, because?.***}

Our approach consists of two phases. First, an exploratory interview phase with software practitioners who have experience working in team environments. Our goal was to understand the difficulties that developers have when they face merge conflicts and in the conflict resolution process. We followed this with a validation survey to get a wider review of our interview findings. 

\subsection{Interviews}
\begin{table*}[!t]
\renewcommand{\arraystretch}{1.3}
\caption{Participant Project Types}
\label{project_type}
\centering
\begin{tabular}{|c|l|c|l|c|}
	\hline
	Partic. & Role & Exp. & Industry & Source Distribution Model\\
	\hline
	P1 & Sr. Software Developer & 18 yrs. & Semiconductor Manuf. & Open-Source\\
	P2 & Software Engineer & 6 yrs. & Semiconductor Manuf. & Open-Source\\
	P3 & Software Engineer & 3 yrs. & Semiconductor Manuf. & Open-Source\\
	P4 & Software Developer & 10 yrs. & Academia & Open-Source\\
	P5 & Infrastructure Engineer & 3 yrs. & Healthcare Software & Closed-Source\\
	P6 & Software Developer & 5 yrs. & Healthcare Software & Closed-Source\\
	P7 & Software Engineer & 5 yrs. & Business Software & Open-Source\\
	P8 & Director & 25 yrs. & Academia & Open-Source\\
	P9 & Collaborator Developer & 8 yrs. & IT Services & Open-Source\\
	P10 & Collaborator Developer & 2 yrs. & Sports \& Wellness Tech. & Open-Source\\
	\hline
\end{tabular}
\end{table*}

\textbf{***interview protocol***}
\comment{AS: need following information to be updated: 1) a demographics table for the interview participants. 2) how many P were from Industry and how many from OSS?}

We conducted 10 semi-structured interviews with software practitioners. Participants were selected from industry contacts or by emailing contributors of top open-source projects. We recruited 6 participants from industry and 4 from open source; participants had a median experience of 5 years (see Table XX). Participants were offered \$50 for their participation.  

Interviews lasted between 30-60 minutes, and were conducted by the first author. Interviews were audio-taped and later transcribed for analysis. At the beginning of the interview we gave participants a short explanation of the research goals and collected demographics data. Participants were then asked about the roles they play in their project, their experience working in team settings, and questions about merge conflicts, the process of conflict resolution, and the difficulties that they faced in conflict resolution. The semi-structured interview format allowed us participants to provide us with unanticipated information. Further, we allowed open ended discussion about merge conflicts in general at the end of the interview, which allowed participants to share ideas that they found particularly important or about topics that we had missed. 

\comment{***analysis process steps***}

Interviews were unitized \cite{unitization} by the first author. Unitization was done by identifying consecutive lines in the interview that seemed logically consistent and grouping them together as one card. This was done in a conservative manner to ensure that context was well preserved within each card. Some further unitization was done during card sorting stage, when both sorters believed that the card needed to be split.

\comment{***why card sorting and process***}

The first two authors individually coded emerging themes in the cards (interview transcripts) and discussed the emerging taxonomies until consensus was reached \cite{card_sort}. We then performed two iterations of card sorting at the end of which we arrived at: the factors that lead to merge conflicts (Section X), the different phases of conflict resolution (Section 3-stages), and the difficulties faced by developers (Section XX4.2).

%Cards were initially sorted into three categories that corresponded to our research questions before following the typical card sorting workflow [blog cited by Zimmerman in 145 Q's paper]. This workflow requires two people to go through each card and group them by themes. The themes can then be divided into sub-groups or generalized into larger groups.

\subsection{Survey}
\todo{cite and discuss \cite{hidden_dependencies} for hidden dependencies option, if needed}

The interview results formed the basis of the survey questions. The survey was divided into four categories. First, elicited background information about demographics, roles, and experience. The following categories reflected the three main interview findings. Each category included about 5-7 questions (see \cite{companion_site} for questions). Most questions used a 5-point Likert scale and included an optional open-ended text form to get additional insights into the questions. 

\comment{***how survey was deployed***}
Survey participants were recruited by mining email addresses from open-source repositories on GitHub, posting to social media sites such as Facebook and Reddit, and by contacting software developers through email. We received 226 survey responses.

\comment{***Analysis***}
For each question, 