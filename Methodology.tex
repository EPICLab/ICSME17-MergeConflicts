\section{Methodology}\label{methodology}
\textbf{***We did a 2-step approach, because?.***}

Our approach consists of two phases. First, an exploratory interview phase with software practitioners who have experience working in team environments. Our goal was to understand the difficulties that developers have when they face merge conflicts and in the conflict resolution process. We followed this with a validation survey to get a wider review of our interview findings. 

\subsection{Interviews}

\begin{table*}[!t]
\renewcommand{\arraystretch}{1.3}
\caption{Interview Participant Demographics}
\label{interview_demographics}
\centering
\begin{tabular}{|c|l|c|c|l|c|c|}
	\hline
	\textbf{Partic.} & \textbf{Role} & \textbf{Exp.} & \textbf{Gender} & \textbf{Industry} & \textbf{Project Model} & \textbf{Project Size}\\
	\hline
	P1 & Sr. Software Developer & 18 yrs. & Male & Semiconductor Manuf. & Open-Source & Large\\
	P2 & Software Engineer & 6 yrs. & Male & Semiconductor Manuf. & Open-Source & Large\\
	P3 & Software Engineer & 3 yrs. & Male & Semiconductor Manuf. & Open-Source & Large\\
	P4 & Software Developer & 10 yrs. & Male & Academia & Open-Source & Small\\
	P5 & Infrastructure Engineer & 3 yrs. & Male & Healthcare Software & Closed-Source & Medium\\
	P6 & Software Developer & 5 yrs. & Male & Healthcare Software & Closed-Source & Medium\\
	P7 & Software Engineer & 5 yrs. & Male & Business Software & Open-Source & Medium\\
	P8 & Director & 25 yrs. & Male & Academia & Open-Source & Large\\
	P9 & Collaborator Developer & 8 yrs. & Male & IT Services & Open-Source & Large\\
	P10 & Collaborator Developer & 2 yrs. & Male & Sports \& Wellness Tech. & Open-Source & Small\\
	\hline
\end{tabular}
\end{table*}

We conducted semi-structured interviews with software practitioners to discover which factors and concerns they care about when working with merge conflicts.
Interview participants were selected from top contributors to open-source projects and from industry contacts using \textit{snowball sampling}~\cite{goodman1961snowball}.

Interviews lasted between 30-60 minutes, and were conducted by the first author. 
Interviews were audio-taped and later transcribed for analysis. 
At the beginning of the interview we gave participants a short explanation of the research goals and collected demographics data. 
Participants were then asked about the roles they play in their project, their experience working in team settings, and questions about merge conflicts, the process of conflict resolution, and the difficulties that they faced in conflict resolution. 

The semi-structured interview format allowed us participants to provide us with unanticipated information~\cite{seaman2008qualitative}. Further, we allowed open ended discussion about merge conflicts in general at the end of the interview, which allowed participants to share ideas and topics that they found particularly important. 

In total, we interviewed 10 participants (see Table~\ref{interview_demographics}) from six different industries: \textit{Semiconductor Manufacturing} (3 participants), \textit{Healthcare Software} (2), \textit{Academia} (2), \textit{Business Software} (1), \textit{IT Services} (1), and \textit{Sports \& Wellness Technology} (1). The median software development experience was 5 years, and participants work on a variety of software project sizes; 5 work primarily on large projects, 3 on medium projects, and 2 on small projects. Participants were offered \$50 for their participation in the interviews.\\

\comment{***analysis process steps***}

Interviews were unitized \cite{unitization} by the first author. Unitization was done by identifying consecutive lines in the interview that seemed logically consistent and grouping them together as one card. This was done in a conservative manner to ensure that context was well preserved within each card. Some further unitization was done during card sorting stage, when both sorters believed that the card needed to be split.

\comment{***why card sorting and process***}

The first two authors individually coded emerging themes in the cards (interview transcripts) and discussed the emerging taxonomies until consensus was reached \cite{card_sort}. We then performed two iterations of card sorting at the end of which we arrived at: the factors that lead to merge conflicts (Section X), the different phases of conflict resolution (Section 3-stages), and the difficulties faced by developers (Section XX4.2).

%Cards were initially sorted into three categories that corresponded to our research questions before following the typical card sorting workflow [blog cited by Zimmerman in 145 Q's paper]. This workflow requires two people to go through each card and group them by themes. The themes can then be divided into sub-groups or generalized into larger groups.

\subsection{Survey}

\begin{table*}[!t]
\renewcommand{\arraystretch}{1.3}
\caption{Survey Participant Roles}
\label{survey_roles}
\centering
\begin{tabular}{|r|c|c|c|c|c|c|c|}
	& \begin{rotate}{45} Software Developers \end{rotate} 
	& \begin{rotate}{45} System Architects \end{rotate} 
	& \begin{rotate}{45} DevOps \end{rotate} 
	& \begin{rotate}{45} Project Managers \end{rotate}
	& \begin{rotate}{45} Project Maintainers \end{rotate}
	& \begin{rotate}{45} Systems Administrators \end{rotate}
	& \begin{rotate}{45} Other \end{rotate}\\
	\hline
	Software Developers & 213 & & & & & & \\
	\hline
	System Architects & 68 & 69 & & & & & \\
	\hline
	DevOps & 64 & 39 & 68 & & & & \\
	\hline
	Project Managers & 53 & 32 & 21 & 54 & & & \\
	\hline
	Project Maintainers & 49 & 27 & 27 & 25 & 50 & & \\
	\hline
	Systems Administrators & 29 & 22 & 21 & 16 & 15 & 31 & \\
	\hline
	Other & 9 & 5 & 4 & 4 & 1 & 2 & 11 \\
	\hline
\end{tabular}
\end{table*}

\todo{Survey participant demographics}
\todo{Find a flower venn diagram for 6}
\todo{cite and discuss \cite{hidden_dependencies} for hidden dependencies option, if needed}

The interview results formed the basis of the survey questions. The survey was divided into four categories. First, elicited background information about demographics, roles, and experience. The following categories reflected the three main interview findings. Each category included about 5-7 questions (see \cite{companion_site} for questions). Most questions used a 5-point Likert scale and included an optional open-ended text form to get additional insights into the questions. 

\comment{***how survey was deployed***}
Survey participants were recruited by mining email addresses from open-source repositories on GitHub, posting to social media sites such as Facebook and Reddit, and by contacting software developers through email. We received 226 survey responses.

\subsubsection{Survey Analysis}
We evaluated the distribution of survey answers for each Likert scale question by analyzing across demographic categories. 
We used the mean score for each demographic to determine the positive, negative, or neutral response of the population. 
We consider the third option (on a 5-option Likert scale) to be neutral, and round scores in the range 2.50 to 3.50 to be equivalent to this neutral response.
