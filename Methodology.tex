\section{Methodology}\label{methodology}

\begin{table}[!t]
\renewcommand{\arraystretch}{1.3}
\captionsetup{justification=centering}
\caption{Interview Participant Demographics\\ \small Project Sizes: (S)mall, (M)edium, (L)arge}
\label{interview_demographics}
\centering
\begin{tabularx}{0.45\textwidth}{@{}clc*{2}{C}@{}}
\toprule
	\textbf{Participant} & \textbf{Role} & \textbf{Exp.} & \textbf{Project Source} & \textbf{Project Size}\\
\midrule
	P1 & Senior \mbox{Software} \mbox{Engineer} & 18 yrs. & Open & L\\
	P2 & Software \mbox{Engineer} & 6 yrs. & Open & L\\
	P3 & Software \mbox{Engineer} & 3 yrs. & Open & L\\
	P4 & Software \mbox{Engineer} & 10 yrs. & Open & S\\
	P5 & Infrastructure \mbox{Engineer} & 3 yrs. & Closed & M\\
	P6 & Software \mbox{Engineer} & 5 yrs. & Closed & M\\
	P7 & Software \mbox{Engineer} & 5 yrs. & Open & M\\
	P8 & Director & 25 yrs. & Open & L\\
	P9 & Collaborator \mbox{Developer} & 8 yrs. & Open & L\\
	P10 & Collaborator \mbox{Developer} & 2 yrs. & Open & S\\
\bottomrule
\end{tabularx}
\end{table}

Our approach consists of two phases - first an exploratory interview phase, followed by a validation survey. We conducted the interviews with software practitioners who have experience working in team environments. Our goal was to understand the difficulties that developers have when they face merge conflicts and in the conflict resolution process. We followed this with a validation survey to to confirm and broaden our interview findings.
This two phase approach allowed us to explore the topic of merge conflicts from an individual developer perspective, and to analyze the broader impact of these perspectives on the larger developer population.

\subsection{Interviews}\label{interview_methods}

We conducted semi-structured interviews with software practitioners to discover which factors and concerns they care about when working with merge conflicts.
Interview participants were selected from top contributors to open-source projects and from industry contacts using snowball sampling~\cite{goodman1961snowball}.

Interviews lasted between 30-60 minutes, and were conducted by the first author. 
Interviews were audio-taped and later transcribed for analysis. 
At the beginning of the interview we gave participants a short explanation of the research goals and collected demographics data. 
Participants were then asked about the roles they play in their project, their experience working in team settings,  questions about merge conflicts, the process of conflict resolution, and the difficulties that they faced in conflict resolution. 

The semi-structured interview format allowed participants to provide us with unanticipated information~\cite{seaman2008qualitative}. Further, we allowed open ended discussion about merge conflicts in general at the end of the interview, which allowed participants to share ideas and topics that they found particularly important. We continued interviewing participants until reaching saturation~\cite{fusch2015we}.

\textit{Demographics:} In total, we interviewed 10 participants (see Table~\ref{interview_demographics}) from six different industries: \textit{Semiconductor Manufacturing} (3 participants), \textit{Healthcare Software} (2), \textit{Academia} (2), \textit{Business Software} (1), \textit{IT Services} (1), and \textit{Sports \& Wellness Technology} (1). The median software development experience was 5 years, and participants worked on a variety of software project sizes; 5 worked primarily on large projects, 3 on medium projects, and 2 on small projects. Participants were offered \$50 for their participation in the interviews.

\textit{Analysis:} Interview transcripts were unitized~\cite{unitization} into cards that each contained a single logically consistent statement. This standardized coding scheme was created by the first and second author, and improved to an acceptable point by measure of intercoder agreement.
Intercoder agreement was determined through \textit{negotiated agreement}, and was reached when no further thematic categories could be created and agreed upon by both coders~\cite{garrison2006revisiting}\cite{ritchie2013qualitative}.
The coding scheme dictated that sentences must be consecutive and topically related to be grouped into a single card. Logically connected statements that were separated by other lines were considered to be separate cards, as a conservative measure to preserve context within each card.

Card sorting, a collaborative technique of exploring how people think about a certain topic~\cite{spencer2009card}\cite{card_sort},  allows key concepts and associations to be identified either through open sorting methods (categories are created during the card sorting process) or closed sorting methods (a set of categories is predefined before beginning to card sort).
The card sorting process consists of: deciding upon the topic space, selecting the method (open or closed), gathering the cards, sorting the cards and recording the data, and analyzing the outcomes.
This process is meant to be both collaborative and argumentative in order to explore the details of a selected topic and can be repeated several times in order to reach consensus among coders.

The first two authors individually coded emerging themes in the cards and discussed the emerging taxonomies until consensus was reached~\cite{card_sort}.
We performed two iterations of the open card sorting process, and found three themes within our resulting categories: the factors that impact how developers approach merge conflicts (Section~\ref{RQ1}), the difficulties that developers face when resolving conflicts (Section~\ref{RQ2}), and the impact of developer tools on the resolution process (Section~\ref{RQ3}).

%Cards were initially sorted into three categories that corresponded to our research questions before following the typical card sorting workflow [blog cited by Zimmerman in 145 Q's paper]. This workflow requires two people to go through each card and group them by themes. The themes can then be divided into sub-groups or generalized into larger groups.

\subsection{Survey}\label{survey_methods}

\begin{table}[!]
\renewcommand{\arraystretch}{1.3}
\caption{\textit{Survey Participant Roles}. Each participant was able to choose multiple roles, so each number represents the overlapping participants between two roles. The total number in each role is on the diagonal.}
\label{survey_roles}
\centering
\begin{tabularx}{0.45\textwidth}{@{}r|*{10}{C}c@{}}
\toprule
\addlinespace[5.4em]
	& \begin{rotate}{45} Soft. Developers \end{rotate} 
	& \begin{rotate}{45} Sys. Architects \end{rotate} 
	& \begin{rotate}{45} DevOps \end{rotate} 
	& \begin{rotate}{45} Project Managers \end{rotate}
	& \begin{rotate}{45} Project Maintainers \end{rotate}
	& \begin{rotate}{45} Sys. Admins \end{rotate}
	& \begin{rotate}{45} Other \end{rotate}\\
\midrule
	Software Developers & 213 & & & & & & \\
	System Architects & 68 & 69 & & & & & \\
	DevOps & 64 & 39 & 68 & & & & \\
	Project Managers & 53 & 32 & 21 & 54 & & & \\
	Project Maintainers & 49 & 27 & 27 & 25 & 50 & & \\
	Systems Administrators & 29 & 22 & 21 & 16 & 15 & 31 & \\
	Other & 9 & 5 & 4 & 4 & 1 & 2 & 11 \\
\bottomrule
\end{tabularx}
\end{table}

We conducted a 50-question survey of software development practitioners in order to examine the themes and categories found in the interviews.
The survey was conducted online and anonymity was guaranteed in order to elicit honest responses from participants.
Questions were developed to confirm, extend, and broaden the results from the interviews.
We sought to understand which factors impact developers the most when they encounter and resolve merge conflicts.

Survey participants were recruited from contributor lists on open-source repositories on GitHub\footnote{https://github.com/}, advertised on social networking sites (Facebook\footnote{https://www.facebook.com/} and Reddit\footnote{https://www.reddit.com/}), and by directly contacting software developers via email. Due to the nature of social media, mailing lists and email forwarding, we cannot know how many people saw our survey and did not respond, so we do not give a response rate.
We received 226 survey responses, but individual questions have varying response rates and are reported where appropriate in Section~\ref{results}.

The population of 226 survey respondents were overwhelmingly male (91.89\% overall). A majority of respondents considered themselves to be \textit{Software Engineer/Developer} (94.25\% overall, see Table~\ref{survey_roles}). They had a median software development experience of 6-10 years (33.63\% overall), and worked on project sizes ranging from 2-51+ developers (the median was 2-5 developers, constituting 49.1\%).

The survey was divided into four categories, with each category containing 5-7 questions (see \cite{companion_site} for questions).
First, we elicited background information about demographics, roles, and experience.
Second, we asked questions related to difficulties that developers experience when encountering merge conflicts.
Third, we asked questions related to conflict resolutions and the factors that affect developers.
Finally we asked questions about the tools and tool features that developers use when working with merge conflicts.
Most questions used a 5-point Likert scale and included an optional open-ended text form to gather additional insights into the questions. 

\textit{Analysis:} We evaluated the distribution of survey answers for each Likert scale question by analyzing across demographic categories. 
We use the mean score for each demographic to determine the positive, negative, or neutral response of the population. 
We considered the third option of the 5-point Likert scale to be moderate, and considered mean scores in the range 2.50 to 3.50 to be equivalent to this moderate response. Mean scores of 3.50 or greater were considered on the higher end of the defined spectrum, while scores less than 2.5 were considered to be on the lower end of the spectrum. This threshold did place a large majority of our factors (76\%) into the moderate category, making our estimates conservative.

Answer choices measured the extent to which participants agreed with a particular answer choice (i.e. \textit{Not at all} to \textit{A great deal}). Because of this, lower mean and median values indicate less agreement than higher mean and median values. As a result, a value of 1.5 and a value of 5 do not represent opposing opinions that we would see in a Likert scale that gives options from Strongly disagree to Strongly agree.