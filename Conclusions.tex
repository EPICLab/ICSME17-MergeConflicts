\section{Conclusion}\label{conclusion}
Practitioner perceptions of merge conflicts have an impact on their development process. First, they use perceptions to determine which tactics they will use to resolve the conflict.
After choosing how to resolve the conflict, practitioners encounter a new set of needs, both technical and social. 
Understanding these perceptions and needs is critical to understanding how to design tools which conform to the issues that these practitioners face in collaborative development.
We provide actionable implications for researchers, tool builders, and practitioners to harness the results of our study.
In future work, we hope to explore whether these factors, needs, and desired toolset improvements can be seamlessly merged into tools or techniques that assist developers' workflows.

%Merge conflicts interrupt development flow.
%Our work explores the human perceptions of merge conflict resolution in version control systems. We conducted an analysis of 10 interviews and confirmed our findings with a survey of 162 practitioners.
%
%We asked how practitioners approach merge conflicts and found that they identify 8 factors of importance for the initial evaluation of merge conflict difficulty. 
%We then asked which factors impact the difficulty of a merge conflict resolution and found 10 factors that impact the difficulty of the merge conflict resolution process. 
%These sets of factors serve as a way to inform managers and future research about merge-conflict-related difficulties of software practitioners.
%
%We also asked if developer tools met practitioner needs for merge conflict resolutions and found 6 improvements which practitioners would like to see in these tools. 
%In addition, we share specific tool needs from our interviews to give an idea of what people really struggle with in their merge conflict resolution workflows.